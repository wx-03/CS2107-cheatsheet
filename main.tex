\documentclass[landscape,a4paper]{extarticle}
\usepackage{caption}
\usepackage{multicol}
\usepackage[top=1em,bottom=1em,left=1em,right=1em]{geometry}
\usepackage[framemethod=tikz]{mdframed}
\usepackage{microtype}
\usepackage{pdfpages}
\usepackage{amsmath, amssymb, amsthm}
\usepackage{anyfontsize}
\usepackage[shortlabels]{enumitem}
\usepackage{graphicx, float}
\usepackage{ulem} % Using \uline{} instead of \underline{} will make the line closer to the word
\usepackage{xcolor}
\usepackage{tgschola}
\usepackage{etoolbox}

% Remove vertical space before and after all verbatim environments
\BeforeBeginEnvironment{verbatim}{\vspace{-\topsep}\vspace{-\partopsep}}
\AfterEndEnvironment{verbatim}{\vspace{-\topsep}\vspace{-\partopsep}}

\let\bar\overline

% Configure image directory
\graphicspath{{images/}}

\newenvironment{Figure}
  {\par\noindent\minipage{\linewidth}}
  {\endminipage\par\medskip}
  
% Remove caption labels
\captionsetup{labelformat=empty,labelsep=none}

% No paragraph indent
\setlength{\parindent}{0pt}

% No spaces between list items, no left margins
\setlist[enumerate]{nosep, leftmargin=*}
\setlist[itemize]{nosep, leftmargin=*}

% No spaces before and after math mode
\expandafter\def\expandafter\normalsize\expandafter{%
    \normalsize%
    \setlength\abovedisplayskip{0pt}%
    \setlength\belowdisplayskip{0pt}%
    \setlength\abovedisplayshortskip{-8pt}%
    \setlength\belowdisplayshortskip{2pt}%
}

\setlength{\columnsep}{6pt}

\begin{document}
\fontsize{5}{5}\selectfont
\fontfamily{qcs}\selectfont
\begin{multicols*}{5}
	\textbf{\uline{Topic 0}}

	\textbf{CIA}
	\begin{enumerate}
		\item Confidentiality
		      \begin{itemize}
			      \item Prevention of unauthorized discolusre of information
		      \end{itemize}
		\item Integrity
		      \begin{itemize}
			      \item Prevention of unauthorized modification of information or processes
			      \item Non-repudiation
			      \item Authentication
		      \end{itemize}
		\item Availability
		      \begin{itemize}
			      \item Prevention of unauthorized withholding of information or resources
		      \end{itemize}
	\end{enumerate}
	\textbf{Threat model}

	\begin{itemize}
		\item The attacker's goals
		\item The attacker's capabilities
	\end{itemize}

	\textbf{Trade-off in security}
	\begin{itemize}
		\item Ease-of-use
		\item Performance
		\item Cost
	\end{itemize}

	\textbf{Threat-Vulnerability-Control}
	\begin{itemize}
		\item \textbf{Threat}: A set of circumstances that has the potential to cause harm (e.g. an
        attacker with control of the workstation in the LT could maliciously gather
        sensitive info like passwords)
        \item \textbf{Vulnerability}: A weakness in the system (e.g. anyone can reboot the
        workstation from USB or Disk to gain control)
        \item \textbf{Control}: A control, countermeasure, security mechanism is a mean to counter threats
        (e.g. restrict physical access to the workstation, disable USB booting)
        \item \textbf{A threat is blocked by control of a vulnerability}
	\end{itemize}
    
    \textbf{\uline{Topic 1: Encryption}}

    \textbf{1.1 Definition: Encryption/decryption/keys}

    \begin{Figure}
        \centering
        \includegraphics[width=\linewidth]{symmetric_key_encryption.png}
    \end{Figure}

    \begin{itemize}
        \item A symmetric-key encryption scheme consists of encryption and decryption
        \item A cipher must be correct and secure
        \begin{itemize}
            \item \textbf{Correctness}: For any plaintext $x$ and key $k$, $D_k(E_k(x)) = x$
            \item \textbf{Security}: Definition depends on the threat models. Informally,
            from the ciphertext, the eavesdropper is unable to derive useful information of the
            key $k$ or the plaintext $x$, even if the eavesdropper can probe the system.
        \end{itemize}
        \item Probabilistic encryption: for the same $x$, there could be different $c$'s.
        But they all can be decrypted to the same $x$.
    \end{itemize}


    \textbf{1.2 Security Model and Requirement}
    
    \textbf{Threat model}
    \begin{itemize}
        \item Attacker's goal
        \begin{itemize}
            \item Total break (most difficult goal)
            \begin{itemize}
                \item Attacker wants to find the key
            \end{itemize}
            \item Partial break
            \begin{itemize}
                \item Attacker may want to decrypt a ciphertext but not interested in knowing the key
                \item Attacker may simply want to extract some info abt the plaintext (e.g. if it is a jpg or excel file)
            \end{itemize}
            \item Distinguishability (weakest goal)
            \begin{itemize}
                \item With some non-negligible probability of $>$ 1/2, the attacker acan correctly
                distinguish the ciphertexts of a given plaintext from the ciphertext of another given
                plaintext 
            \end{itemize}
        \end{itemize}
        \item Attacker's capability
        \begin{itemize}
            \item Ciphertext only attack
            \begin{itemize}
                \item Attacker is given a collection of ciphertext $c$. The attacker may 
                know some properties of the plaintext (e.g. the plaintext is an English sentence)
            \end{itemize}
            \item Known plaintext attack
            \begin{itemize}
                \item The attacker is given a collection of plaintext $m$ and their corresponding
                ciphertext $c$
                \item Attacker might get this as they know the header or part of the plaintext
            \end{itemize}
            \item Chosen plaintext attack (CPA)
            \begin{itemize}
                \item The attacker has access to an oracle. The attacker can choose and feed any plaintext
                $m$ to the oracle and obtain the corresponding ciphertext $c$ (all encrypted with
                the same key). The attacker can access the oracle many times, as long as within the attacker's
                compute power. He can see the ciphertext and then choose the next input. This black-box is an
                \textbf{encryption oracle}. 
                \item e.g. attacker has access to a smartcard
                \item e.g. attacker can eavesdrop
            \end{itemize}
            \item Chosen ciphertext attack (CCA2)
            \begin{itemize}
                \item Same as CPA but the attacker chooses the ciphertext and the black-box
                outputs the plaintext. The black-box is a \textbf{decryption oracle}.
                \item Padding oracle is a weaker form of a decryption oracle.
            \end{itemize}
        \end{itemize}
    \end{itemize}

    From defender's POV, want a cipher that can protect against the attacker with the highest
    capability. Cipher is secure against CCA2 (decryption oracle) $\implies$ secure
    against CPA (encryption oracle)

    \textbf{1.3 Classical ciphers + illustration of attacks}

    \textbf{1.3.1 Substitution cipher}

    \begin{itemize}
        \item Plaintext and ciphertext are both strings over a set of symbols $U$.
        \item The key is a 1-1 onto func from $U$ to $U$
        \item Key space: set of all possible keys
        \item Key space size: total number of possible keys
        \item Key size/length: number of bits required to represent a key
        \item Attacks
        \begin{enumerate}
            \item Exhaustive search (examine all possible keys 1 by 1)
            \begin{itemize}
                \item Running time depends on size of key space
                \item If the table size is 27, the key can be represented by a sequence of 27
                symbols. The size of key space is 27!. Exhaustive search eneds to carry out 27!
                loops, which is infeasible using current compute power.
            \end{itemize}
            \item Known plaintext attack
            \begin{itemize}
                \item Given sufficiently long ciphertext, the full table can be found
                \item Substitution cipher is not secure under known plaintext attack.
            \end{itemize}
            \item Ciphertext only attack
            \begin{itemize}
                \item Given that the attacker knows that the plaintext is an English sentence,
                he can do frequency analysis attack. The frequency of letters used in English is 
                not uniform. Given a sufficiently long ciphertext, attacker may correctly guess the plaintext by 
                mapping frequent characters in the ciphertext to the frequent character in English.
            \end{itemize}
        \end{enumerate}
    \end{itemize}

    \textbf{1.3.2 Permutation cipher}
    \begin{Figure}
        \centering
        \includegraphics[width=\linewidth]{permutation_cipher.png}        
    \end{Figure}
    \begin{itemize}
        \item AKA transposition cipher
        \item First group the plaintext into blocks of $t$ characters, then apply a secret
        permutation to each block by shuffling the characters
        \item The key is the secret permutation, which is a 1-1 onto func $e$ from $\{1, 2, \ldots, t\}$
        to $\{1, 2, \ldots, t\}$. $t$ can also be part of the key.
        \item Attack
        \begin{itemize}
            \item Fails under known-plaintext attack
            \item Easily broken under ciphertext only attack if the plaintext is English text
        \end{itemize}
    \end{itemize}

    \textbf{1.3.3 One Time Pad}

    \textbf{Properties of xor}: 
    \begin{itemize}
        \item Commutative: $A \oplus B = B \oplus A$
        \item Associative: $A \oplus (B \oplus C) = (A \oplus B) \oplus C$
        \item Identity element: $A \oplus 0 = A$
        \item Self-inverse: $A \oplus A = 0$
    \end{itemize}

    \textbf{One Time Pad}
    \begin{itemize}
        \item Encryption: plaintext xor key bit by bit
        \item Decryption: ciphertext xor key bit by bit
        \item Key is only used once, so 1GB of plaintext would need a 1GB key to encrypt
        \item Security
        \begin{itemize}
            \item From a pair of ciphertext and plaintext, attacker can derive the key
            but useless bc key won't be used anymore
        \end{itemize}
    \end{itemize}

    \textbf{1.4 Modern ciphers + recommended key length}

    \textbf{1.4.2 Block cipher \& mode of operations}

    DES/AES are known as block ciphers. Block ciphers have a fixed size of input/output.
    AES: 128 bits (16 bytes). 

    Large plaintext is divided into blocks before applying the block cipher.

    \textbf{ECB (electronic code book) mode}
    \begin{Figure}
        \centering
        \includegraphics[width=\linewidth]{ecb_encryption.png}        
    \end{Figure}
    \begin{Figure}
        \centering
        \includegraphics[width=\linewidth]{ecb_penguin.png}        
    \end{Figure}

    \textbf{CBC (cipher block chaining) mode on AES}
    \begin{itemize}
        \item Initialization vector (IV) is an arbitrary value chosen during encryption, 
        must be different in different encryptions. 
        \item In CBC mode, IV must be unpredictable, else it is susceptable to BEAST attack.
        \item If IV is randomly chosen, it is unpredictable
    \end{itemize}
    \begin{Figure}
        \centering
        \includegraphics[width=\linewidth]{cbc_encryption.png}        
    \end{Figure}
    \begin{Figure}
        \centering
        \includegraphics[width=\linewidth]{cbc_decryption.png}        
    \end{Figure}

    \textbf{CTR (counter) mode}
    \begin{Figure}
        \centering
        \includegraphics[width=\linewidth]{ctr_encryption.png}        
    \end{Figure}
    \begin{Figure}
        \centering
        \includegraphics[width=\linewidth]{ctr_decryption.png}        
    \end{Figure}

    \textbf{GCM mode (Galois/counter)}

    Authenticated encryption, ciphertext consists of extra tag for authentication.
    Secure in the presence of decryption oracle.

    \textbf{1.4.3 Stream cipher and IVs}

    Stream cipher is bit by bit. CTR mode is a ``stream cipher'' but it is not bit by bit.
    \begin{Figure}
        \centering
        \includegraphics[width=\linewidth]{stream_cipher.png}        
    \end{Figure}

    \begin{itemize}
        \item Need IV and no two IVs can be the same
    \end{itemize}

    \textbf{Stream cipher without IV}
    \begin{Figure}
        \centering
        \includegraphics[width=\linewidth]{stream_cipher_without_iv.png}        
    \end{Figure}

    \textbf{Stream cipher with IV}
    \begin{Figure}
        \centering
        \includegraphics[width=\linewidth]{stream_cipher_with_iv.png}        
    \end{Figure}

    \begin{itemize}
        \item IV makes an encryption probabilistic
    \end{itemize}
    
    \textbf{1.5 Examples of attacks on crypto}

    \textbf{1.5.1 Meet-in-the-middle}

    \begin{itemize}
        \item DES is not secure $\rightarrow$ improve by encrypting multiple times using different
        keys
        \item Consider double encryption under known plaintext attack. Attacker has $m$ and $c$ and
        wants to know $k_1$, $k_2$.
        \item Using exhaustive search, amount of DES encryption/decryption would be $2^{56+56}$
        \item Hence use meet-in-the-middle attack.
        % Compute sets $C$ and $M$, where $C$ contains
        % ciphertexts of $m$ encrypted with all possible keys and $M$ contains plaintexts of $c$ decrypted
        % with all possible keys. Then find all common elements (likely only 1) in $C$ and $M$. From 
        % common elements, obtain the 2 keys.
        \item for $k$-bit keys, this reduces the number of crypto operations to $2^{k + 1}$
    \end{itemize}

    \begin{Figure}
        \centering
        \includegraphics[width=\linewidth]{meet_in_the_middle.png}        
    \end{Figure}

    \textbf{Tradeoff with time and space}

    \begin{Figure}
        \centering
        \includegraphics[width=\linewidth]{meet_in_the_middle_tradeoff.png}        
    \end{Figure}

    \begin{itemize}
        \item Last bit of $k_1$ fixed to 1, last bit of $k_2$ fixed to 0
        \item Perform meet-in-the-middle on the first 2 bits of $k_1$ and $k_2$
    \end{itemize}

    \textbf{1.5.2 Padding Oracle}

    Plaintext needs to be padded to split into blocks

    \begin{itemize}
        \item PKCS\#7 is a padding standard
    \end{itemize}

    \begin{Figure}
        \includegraphics[width=0.5\linewidth]{pkcs7.png}        
    \end{Figure}

    \textbf{Padding oracle attack}

    % Attack model: 

    % Attacker has:
    % \begin{enumerate}
    %     \item Ciphertext (iv, c) where the ciphertext was encrypted using $k$
    %     \item Access to a padding oracle
    % \end{enumerate}
    % Attacker's goal: plaintext of (iv, c)

    % Padding oracle input is ciphertext, output is YES if the plaintext is in the correct
    % padding format else NO

    \textbf{Padding oracle attack on AES CBC mode}

    Attacker knows:

    \begin{Figure}
        \centering
        \includegraphics[width=\linewidth]{padding_oracle_attacker_knowledge.png}
    \end{Figure}

    \begin{Figure}
        \centering
        \includegraphics[width=\linewidth]{padding_oracle_attack.png}
    \end{Figure}

    \begin{align*}
        iv \oplus d(c) &= 03\\
        iv' \oplus d(c) &= 04
    \end{align*}

    xor the 2 tgt to get $iv' = 07 \oplus iv$

    \begin{align*}
        iv' &= iv \oplus 00\ 00 \ldots t\ 07\ 07\ 07\\
        d(C_5) \oplus t \oplus V_5 &= 04\\
        d(C_5) \oplus V_5 \oplus t &= 04\\
        d(C_5) \oplus V_5 &= x_5\\
        x_5 \oplus t &= 04\\
        x_5 &= 04 \oplus t
    \end{align*}
    
    Keep guessing $t$ until padding oracle outputs YES, then we know $x_5$

    To get next byte:
    \begin{Figure}
        \centering
        \includegraphics[width=\linewidth]{padding_oracle_next_byte.jpg}
    \end{Figure} 

    \textbf{1.6 Pitfalls in usages and implementations}
    \begin{enumerate}
        \item Wrong choice of IV / reusing one-time pad
        \item Randomness is predictable
        \item Modify existing or design your own encryption scheme
        \item Reliance on obscurity: Kerkchoff's principle
        \begin{itemize}
            \item Kerkchoff's principle: A system should be secure ven if everything about the
            system, except the secret key, is public knowledge
        \end{itemize}
    \end{enumerate}

    \textbf{\uline{Topic 2: Authentication Credential}}

    \textbf{Authentication}

    Authentication is the process of assuring that the communicating identity, or origin of a
    piece of information, is the one that it claims to be.

    \begin{itemize}
        \item Authentication implies integrity.
    \end{itemize}

    \textbf{Data-origin authentication}: is a piece of data generated by an authentic entity?
    \begin{itemize}
        \item Signature or MAC (message authentication code)
    \end{itemize}

    \textbf{Commnication authentication}: is the enitity interacting with the verifier an authentic entity?
    \begin{itemize}
        \item Authentication protocol
    \end{itemize}

    \textbf{2.2 Password}

    \textbf{Password vs key}

    Passwords are generated by human and can be remembered by human. Keys are binary sequences that are infeasible to be 
    remembered by humans.

    \textbf{Password system}

    \begin{enumerate}
        \item Bootstrapping
        \begin{itemize}
            \item User and server establish a common password, server keeps a password file keeping the 
            identity and the corresponding password
            \item Password established during boostrapping either by a default password
            or by the server/user choosing a password and sending it to the user/server through another
            communication channel
        \end{itemize}
        \item Authentication
        \begin{itemize}
            \item Server authenticates an entity. An entity who can convince the server that it knows
            the password is deemed to be authentic.
        \end{itemize}
        \item Password reset
        \begin{itemize}
            \item Need to authenticate the entity before allowing the entity to change password.
            \item Need another credential (other than the old password) to authenticate bc ppl might 
            want to reset password when they forget
            \item Can be done through OTP, security question (not secure as entropy of answers is 
            lower than the password)
        \end{itemize}
    \end{enumerate}

    \textbf{Attack on passwords}

    \textbf{2.2.1 Attack on Bootstrapping}
    \begin{itemize}
        \item Attacker intercepts the password during bootstrapping, e.g. if password
        is sent through postal mail, an attacker steals the mail to get the password
        \item Attacker uses the "default" passwords
        \begin{itemize}
            \item Mitigation: require the user to change password after first login
        \end{itemize}
        \item Example: zoom flaw allowed account hijacking
    \end{itemize}

    \textbf{2.2.2 Attack on Password Reset}
    \begin{itemize}
        \item Mechanism of security questions weakens the password system, but it is less
        common now
        \item Social engineering + password reset
    \end{itemize}

    % \begin{Figure}
    %     \centering
    %     \includegraphics[width=0.9\linewidth]{social_engineering_pw_reset.png}
    % \end{Figure}

    \textbf{2.2.3 Searching for the password}

    \textbf{Dictionary attacks}
    \begin{itemize}
        \item Test passwords using a dictionary that could contain words from English dict, known
        compromised passwords, etc.
        \item Also test combinations of words in the dictionary. e.g. combinations of 2 words, all possible capitalizations
        of letters in each word, substituting `a' with `@', etc.
        \item \textbf{Online dictionary attack}
        \begin{itemize}
            \item To test a password, attacker must interact with the authentication system
        \end{itemize}
        \item \textbf{Offline dictionary attack}
        \begin{itemize}
            \item Attacker first obtains some information $D$ about the password, possibly by sniffing the login session
            of an authentic user, or by interacting with the server. (e.g. attacker obtains the hashed password)
            \item Next, the user carries out dictionary attack using $D$ without interacting with the system (e.g. attacker
            compares the hashed password with the hashed words in dictionary)
        \end{itemize}
        \item Guessing password from social information
    \end{itemize}

    \textbf{2.2.4 Stealing the password}
    \begin{enumerate}
        \item Sniffing
        \begin{itemize}
            \item Shoulder surfing: look-over-the-shoulder attack
            \item Sniffing the communication: Some systems simply send the password over
            the public network in clear (i.e. not encrypted), e.g. FTP, Telnet, HTTP
            \item Sniff wireless keyboards that employ insecure encryption method
            \item Using sound made by keyboard
            \item Viruses, key-logger
            \begin{itemize}
                \item Key-logger captures keystrokes and sends the info back to the attacker.
                \item Can be in the form of software (viruses) or hardware.
            \end{itemize}
        \end{itemize}
        \item Phishing
        \begin{itemize}
            \item Victim is tricked into voluntarily sending the password to the attacker
            \item Asks for password under false pretense
        \end{itemize}
        \item Spear Phishing
        \begin{itemize}
            \item Phishing that is targeted to a particular small group of users, e.g. NUS staff
        \end{itemize}
    \end{enumerate}

    \textbf{Phishing Prevention}
    \begin{itemize}
        \item User training
        \item Blacklisting, e.g. phishtank.com
        \item Visually spot by ensuring that there is a padlock in the address bar and that the 
        domain name in the url is correct
    \end{itemize}

    \textbf{2.2.5 Password strength}
    \begin{itemize}
        \item We quantify the key-strength by the size of the key if best-known attack is exhaustive search.
        \item If best-known attack is faster, then we quanitfy it by its equivalent in exhaustive search.
    \end{itemize}

    \textbf{Using strong password}
    \begin{itemize}
        \item Truly random password: password is chosen randomly among all possible keys using an automated passsword
        generator. High entropy but difficult to remember.
        \item User selection: 
        \begin{itemize}
            \item Mnemonic method
            \item Altered passphrases
            \item Combining and altering word
        \end{itemize}
        \item Usability:
        \begin{itemize}
            \item Strong passwords are difficult to remember
            \item It is difficult to enter alphanumeric passwords into mobile devices. There are alternatives,
            e.g. graphical or gesture-based
        \end{itemize}
    \end{itemize}

    \textbf{Password entropy}

    Suppose a set $P$ contains $N$ unique passwords. A password is chosen by randomly picking
    a password from $P$. Entropy of password is
    \begin{equation*}
        -\sum_{i=1}^N p_i \log_2p_i
    \end{equation*}
    where $p_i$ is the probability that the $i$-th password is picked.

    If the password is chosen uniformly, each password in $P$ has probability of $1/N$ of
    being chosen. The entropy of the password is
    \begin{equation*}
        log_2N \text{ bits}
    \end{equation*}

    For the entropy to be highest for a set of $N$ items, $p_i$ must be $1/N$

    \textbf{Additional protection to password files}

    Password file should be hashed and salted
    \begin{Figure}
        \centering
        \includegraphics[width=\linewidth]{hashed_salted.jpg}        
    \end{Figure}

    Make it harder for rainbow table attack

    \textbf{2.3 Biometric}

    Biometric data is the password

    \begin{align*}
        &\text{FMR (false positive)}\\
       =\ &\frac{\text{no. of successful false matches (B)}}{\text{no. of attempted false matches (B + D)}}
    \end{align*}
    \vspace{1px}
    \begin{align*}
        &\text{FNMR (false negative)}\\
       =\ &\frac{\text{no. of rejected genuine matches (C)}}{\text{no. of attempted genuine matches (A + C)}}
    \end{align*}
    \begin{Figure}
        \centering
        \includegraphics[width=0.8\linewidth]{fmr_fnmr.jpg}        
    \end{Figure}
    Threshold: FNMR/FMR. Lower threshold more relax, higher threshold more stringent

    \textbf{Attack on biometric system}

    Biometric data can be spoofed, use liveness detection e.g. temperature sensor in fingerprint scanner

    \textbf{2.4 n-factor Authentication and Multi-Step Verification}

    \textbf{n-factor Authentication}

    Requires at least 2 different authentication ``factors''

    \begin{enumerate}
        \item Something you know: password, PIN
        \item Something you have: Security token, smart card, phone, ATM card
        \item Who you are: Biometric
    \end{enumerate}

    \textbf{Multi-Step Verification}

    If both are the same category of factors (2 passwords, both are something you know) 
    then it is 2-step verification

    \textbf{\uline{Topic 3: Authenticity (data origin)}}

    \textbf{3.1 PKC}

    \begin{itemize}
        \item With multiple identitites, many pairs of symmetric keys are required.
        \item Symmetric key requires both entities to know each other before the actual
        communication session. Hence use public key encryption.
    \end{itemize}

    \begin{Figure}
        \centering
        \includegraphics[width=\linewidth]{public_key_encryption.png}
    \end{Figure}

    \textbf{Popular PKC schemes}
    \begin{itemize}
        \item RSA
        \item ElGamal
        \item Paillier
        \item Post-quantum cryptography
    \end{itemize}

    \textbf{3.1.1 RSA}

    \begin{enumerate}
        \item Owner randomly chooses 2 large primes $p, q$ and computes $n=pq$
        \item Owner randomly chooses an encryption exponent $e$ s.t. $\gcd(e, (p - 1)(q - 1)) = 1$
        \item Owner finds decryption exponent $d$ where $d e \mod(p - 1)(q - 1) = 1$
        \item Owner publishes $\langle n, e \rangle$ as public key, and safe-keeps $d$ as the private key
    \end{enumerate}

    \begin{Figure}
        \centering
        \includegraphics[width=\linewidth]{textbook_rsa.jpg}        
    \end{Figure}

    Got algo to find $d$ given $e, p, q$. For faster speed, choose small $e$. Common value is 65537.
    $e$ is not a secret in such cases

    \textbf{3.1.2 Security of RSA}

    Getting RSA private key from public key is as difficult as factorizing $n$.

    \textbf{Padding of RSA}

    \begin{itemize}
        \item Some forms of IV is required so that encryption of the same plaintext at different
        times would give different ciphertexts. Additional padding required for security.
    \end{itemize}

    \textbf{3.1.3 Efficiency}

    RSA encryption/decryption is significantly slower than AES. 
    Can use PKC to encrypt a symmetric key then use AES for encryption

    \textbf{3.2 Data Authenticity}

    \textbf{Security requirement of hash}

    \begin{itemize}
        \item Collision-resistant
        \begin{itemize}
            \item Collision: Find 2 different messages $m_1, m_2$ s.t. $h(m_1) = h(m_2)$
        \end{itemize}
        \item 2nd pre-image resistant
        \begin{itemize}
            \item 2nd pre-image: Given $m_1$, find $m_2$ s.t. $h(m_1) = h(m_2)$
        \end{itemize}
        \item One-way
        \begin{itemize}
            \item Pre-image: Given $y$, find $m$ s.t. $h(m) = y$
        \end{itemize}
    \end{itemize}

    \textbf{Application of unkeyed hash}

    When downloading something from a website, match the hash of the file with the checksum
    displayed in the browser.

    \begin{Figure}
        \centering
        \includegraphics[width=\linewidth]{hash_application_vlc.jpg}
    \end{Figure}

    If not 2nd pre-image resistant, can be attacked

    \textbf{3.3 Data Origin Authenticity (mac), keyed}

    Keyed-hash is a function whose input is an arbitrary large message and a secret key, output is a
    fixed-size mac (message authentication code)
    \begin{itemize}
        \item Security requirement (forgery): Even if attacker sees multiple valid pairs of messages and their
        corresponding mac, it is difficult for the attacker to forge the mac of a message not seen before
        \item CBC-MAC: based on AES operated under CBC mode
        \item HMAC: Hashed-based MAC based on SHA
    \end{itemize}

    \textbf{Application of mac}

    Same situation as before but dh secure channel to deliver digest. Protect the digest
    with the help of some secrets.
    \begin{itemize}
        \item In symmetric key setting, called mac
        \item In public key setting, called digital signature
        \item mac typically appended to file, also called authentication tag or authentication code
    \end{itemize}

    \textbf{3.4 Data Origin Authenticity (Signature)}

    Public key version of MAC is called signature
    \begin{itemize}
        \item Owner uses private key to generate signature, public uses public key to verify
        signature
    \end{itemize}

    \begin{Figure}
        \centering
        \includegraphics[width=\linewidth]{signature.jpg}
    \end{Figure}
    Signature is appended to the file F
    \begin{itemize}
        \item Signature scheme achieves \textbf{non-repudiation}
    \end{itemize}
    \begin{Figure}
        \centering
        \includegraphics[width=\linewidth]{signature_generation_verification.jpg}
    \end{Figure}
    hash and sign / hash and encrypt

    \textbf{3.5.1 Birthday Attacks}

    Birthday attack is used to find collision. 

    Suppose we have $M$ messages, and each message is tagged with a value randomly chosen from $\{1,2,3, \ldots, T\}$.
    Then the probability that there is a pair of messages tagged with the same value is approx

    \begin{equation*}
        1-e^{\left(-\frac{M^2}{2T}\right)}
    \end{equation*}

    Let $S$ be a set of $k$ distinct elements where each element is an $n$-bits binary
    string. Let us independently and randomly select $m$ $n$-bit binary strings and put
    them in the set $T$. The probability that $S$ has non-empty intersection with $T$ is
    more than
    \begin{align*}
        1 - 2.7^{-km2^{-n}}
    \end{align*}

    \textbf{\uline{4. PKI + Channel Security}}

    \textbf{4.1 Distribution of public keys}
    \begin{itemize}
        \item PKC requires a secure broadcast channel to distribute public keys: PKI
        \item If no secure way to distribute public key, attacker can impersonate by giving
            his own public key
        \item Certificate: A piece of document that binds a ``name'' to a ``public key'' \&
            certified by a CA
        \item Certificate contains:
        \begin{itemize}
            \item Name, public key, expiry date
            \item Meta info: usages, type of crypto, name of CA, etc
            \item CA's signature
        \end{itemize}
    \end{itemize}

    \textbf{4.2 PKI}

    \textbf{4.2.1 Certificate \& CA}

    \begin{itemize}
        \item Certificates are used to distribute public keys. A CA issues certificates. 
        \item CA: trusted authority that manages a directory of public keys
        \item CA has its own public-private key, some CA's public keys have been distributed securely 
            through other means.
        \item OSes and browsers have pre-loaded CA's public keys, these CAs are known as root CAs.
        \item Other CA's public keys added through chain-of-trust
        \item A \textbf{certificate} is a digital document containing at least the following:
        \begin{itemize}
            \item Name (e.g. alice@yahoo.com / bbc.com / *.bbc.com)
            \item Public key of the owner
            \item Time window that this cert is valid
            \item Signature of CA
        \end{itemize}
    \end{itemize}

    \textbf{4.2.2 CA's chain-of-trust}

    \textbf{Responsibility of CA}

    \begin{itemize}
        \item Issue certificate
        \item Verify that information is correct (by checking that the applicant indeed owns the domain name)
    \end{itemize}

    \textbf{Certificate chain-of-trust}

    \begin{itemize}
        \item If Alice's cert is issued by CA\#1, but Bob doesn't have the public key of CA\#1, then
            Alice can send her email, her cert, and CA\#1's cert (issued by root CA) to Bob
        \item Bob can now:
        \begin{itemize}
            \item Verify CA\#1's certificate using root CA's public key
            \item Verify Alice's certificate using CA\#1's public key
            \item Verify Alice's email using ALice's public key
        \end{itemize}
        \item Bob can also obtain CA\#1's certificate from other sources
    \end{itemize}

    \textbf{4.2.3 Revocation}

    Reasons for revoking non-expired certs:

    \begin{itemize}
        \item Private key compromised (breaches, insider attack, vulnerability in choosing private keys)
        \item System admin left an organization
        \item Business entity closed
        \item Issuing CA was compromised
    \end{itemize}

    Verifier needs to check whether certificate is still valid, even if it has not expired. 2 approaches:

    \begin{itemize}
        \item Certificate Revocation List (CRL)
        \begin{itemize}
            \item CA periodically signs and publishes a revocation list
        \end{itemize}
        \item Online Certificate Status Protocol (OCSP)
        \begin{itemize}
            \item OCSP Responder validates a cert in question. Need online OCSP responder
        \end{itemize}
    \end{itemize}

    Recommendation: User periodically updates its local cache of revocation list, user does not need to check online
    to verify a cert

    \textbf{4.3 Limitations / attacks on PKI}
    \begin{enumerate}
        \item Implementation bugs
        \begin{itemize}
            \item Some browsers ignore substrings in the ``name'' field after the null characters when displaying it 
            in the address bar but include them when verifying the cert.
            \begin{itemize}
                \item Name in cert:\\
                \texttt{www.comp.nus.edu.sg\char`\\0.hacker.com}
                \item Displayed in browser as:\\
                \texttt{www.comp.nus.edu.sg}
            \end{itemize}
        \end{itemize}
        \item Users think they are connected to\\
        \texttt{www.comp.nus.edu.sg}\\
        but are actually connected to\\
        \texttt{www.comp.nus.edu.sg\char`\\0.hacker.com}
        \item Abuse by CA
        \begin{itemize}
            \item Rogue CA can forge any certificate
        \end{itemize}
        \item Social engineering
        \begin{enumerate}
            \item Typosquatting
            \begin{itemize}
                \item An attacker registers for a domain name that looks similar to another website
            \end{itemize}
            \item Sub-domain
            \begin{itemize}
                \item Attacker is the rightful owner of\\
                \texttt{134566.com}
                \item Attcker creates a sub domain\\
                \texttt{luminus.nus.edu.sg.134566.com}
                \item Attacker can get a valid certificate
            \end{itemize}
        \end{enumerate}
    \end{enumerate}

    \textbf{4.4 Protocol 1: Authentication}

    \begin{Figure}
        \centering
        \includegraphics[width=\linewidth]{authentication_summary.png}        
    \end{Figure}

    To prevent replay, use challenge-response

    \begin{Figure}
        \centering
        \includegraphics[width=0.5\linewidth]{challenge_response.png}        
    \end{Figure}

    \begin{itemize}
        \item $m$ is picked at random
        \item $k$ is shared secret between Alice and Bob
        \item Protocol only authenticates Alice. Unilateral authentication.
    \end{itemize}

    Can do unilateral authentication using PKC as well

    \begin{Figure}
        \centering
        \includegraphics[width=0.5\linewidth]{unilateral_authentication_pkc.png}        
    \end{Figure}

    \begin{itemize}
        \item Alice may send Bob certificate if required (if Bob doesn't have Alice's public key)
    \end{itemize}

    If only authentication is done, Mallory can interrupt and take over the channel 
    after Bob is convinced that he is communicating with Alice. Use authenticated key-exchange to get a new 
    shared secret $k$ known as session key. Then all subsequent communication will be encrypted + mac using $k$

    \textbf{4.5 Protocol 2: Key Exchange}

    \begin{Figure}
        \centering
        \includegraphics[width=\linewidth]{key_exchange_summary.png}
    \end{Figure}

    \textbf{PKC-based key exchange}

    \begin{Figure}
        \centering
        \includegraphics[width=\linewidth]{pkc_based_key_exchange.png}
    \end{Figure}

    \textbf{Diffie-Hellman key exchange}

    \begin{Figure}
        \centering
        \includegraphics[width=\linewidth]{diffie_hellman_key_exchange.png}
    \end{Figure}

    \begin{itemize}
        \item Alice and Bob have agreed on $g$ and $p$. They are not secret and known to the public.
        \item Security relies on the CDH (computational diffie-hellman) assumption: given 
        $g$, $p$, $x = g^a \mathbin{\%} p$, it is computationally infeasible to find $k = g^{ab} \mathbin{\%} p$
    \end{itemize}

    \begin{Figure}
        \centering
        \includegraphics[width=\linewidth]{diffie_hellman_example.png}
    \end{Figure}

    Diffie-hellman meets forward secrecy requireent but PKC based method doesn't. TLS 1.3 mandates
    forward secrecy

    \textbf{4.6 Protocol 3: Authenticated Key Exchange}

    Key exchange alone cannot guard against Mallory:

    \begin{Figure}
        \centering
        \includegraphics[width=\linewidth]{mitm_key_exchange.png}
    \end{Figure}

    Need authenticated key-exchange, which can be easily obtained from existing key exchange.

    \begin{itemize}
        \item Diffie Hellman based:
        \begin{itemize}
            \item Sign all communication using private key. Known as station-to-station protocol.
        \end{itemize}
        \item PKC based:
        \begin{itemize}
            \item Omit setep (1) (generating public/private keys) and use Alice's existing public/private
            keys. Since only Alice has her private key, the entity that can correctly decrypt it must be Alice
        \end{itemize}
    \end{itemize}

    \textbf{Station-to-station Protocol}

    \begin{Figure}
        \centering
        \includegraphics[width=\linewidth]{station_to_station.png}
    \end{Figure}

    \textbf{PKC-based Authenticated Key-exchange\\
    (AKA RSA-based)}

    \begin{Figure}
        \centering
        \includegraphics[width=\linewidth]{pkc_based_authenticated_key_exchange.png}
    \end{Figure}

    \textbf{Mutual Authenticated Key exchange}

    \begin{itemize}
        \item Before:
        \begin{itemize}
            \item Alice has a pair of public, private keys (A\textsubscript{public}, A\textsubscript{private})
            \item Bob has a pair of public, private keys (B\textsubscript{public}, B\textsubscript{private})
            \item Alice knows Bob's public key and vice versa. The two sets of keys are known as the long-term key or master key
        \end{itemize}
        \item Carry out authenticated key exchange protocol (e.g. STS). If an entity is not authentic, the other will halt.
        \item After:
        \begin{itemize}
            \item Both A and B obtain shared key $k$, known as session key
        \end{itemize}
        \item Security requirement:
        \begin{itemize}
            \item (Authenticity) Alice is assured that she is communicating with an entity who knows B\textsubscript{private}
            \item (Authenticity) Bob is assured that he is communicating with an entity who knows A\textsubscript{private}
            \item (Confidentiality) Attacker is unable to get the session key
        \end{itemize}
    \end{itemize}

    \textbf{4.7 Securing Communication Channel}
    
    \textbf{TLS} 

    Alice wants to visit \texttt{bob.com}.

    \begin{enumerate}
        \item Bob sends his certificate to Alice.
        \item Alice and \texttt{bob.com} carry out unilateral authenticated key exchange protocol with Bob's public/private
        key. After the protocol, both Bob and Alice obtain $k$, which could come in the form
        of $k = \langle k_1, k_2 \rangle$ whwere $k_1$ is the secret key of the MAC, and $k_2$
        is the secret key of the symmetric-key encryption, or a single key $k$ when authenticated encryption (e.g. GCM) is 
        in use. These keys are called the session keys. By property of the protocol, Alice is convinced that
        only Bob and herself know the session key.
        \item Subsequent interactions between Alice and \texttt{bob.com} will be protected by the session keys and a
        sequence number. Suppose $m_1, m_2, m_3$ are the sequence of message exchanged, 
        the actual data to be sent for $m_i$ is

        \texttt{E\textsubscript{k1}(i || m) || mac\textsubscript{k2}(E\textsubscript{k1}(i || m))} (still in use but not recommended)

        For GCM mode or other authenticated encryptions, the message to be sent is simply

        \texttt{E\textsubscript{k}(i || m)}
    \end{enumerate}

    \begin{Figure}
        \centering
        \includegraphics[width=\linewidth]{TLS.jpg}
    \end{Figure}

    \begin{itemize}
        \item SSL and TLS are protocols that secure communication using cryptographic means
        \item SSL is the predecessor of TLS
        \item HTTPS is built on top of TLS
    \end{itemize}

    \textbf{TLS handshake}

    \begin{Figure}
        \centering
        \includegraphics[width=\linewidth]{TLS_handshake.png}
    \end{Figure}

    \textbf{4.8 Forward Secrecy}

    If Eve cannot recover plaintext even though she knows the master key, then the authenticated
    key exchange achieves forward secrecy

    \begin{itemize}
        \item PKC-based authenticated key exchange does not achiever forward secrecy
        \item Station-to-station key exchange achieves forward secrecy
        \begin{itemize}
            \item If attacker can solve CDH, then forward secrecy of station-to-station is compromised
        \end{itemize}
    \end{itemize}

    \textbf{\uline{5. Network Security}}

    \textbf{5.2 Background: networking}

    Computer network: a collection of interconnected devices that can communicate with one
    another 

    \textbf{Network layers}
    \begin{enumerate}
        \item Physical (wifi, ethernet)
        \item Data link
        \begin{itemize}
            \item MAC address
        \end{itemize}
        \item Network (IP)
        \begin{itemize}
            \item IP address
        \end{itemize}
        \item Transport (TCP, UDP)
        \begin{itemize}
            \item Port number
        \end{itemize}
        \item Application
        \begin{itemize}
            \item Domain name
        \end{itemize}
    \end{enumerate}
    
    \begin{Figure}
        \centering
        \includegraphics[width=\linewidth]{data_flow_layers.png}
    \end{Figure}

    \textbf{Transport + IP layer}
    \begin{itemize}
        \item Often treated as one single layer
        \item Address of a communicating entity is an ip addr and a Port
        \item Each node in the network has 65535 ports
        \item Communication channel between 2 nodes is established by connecting 2 ports
        \item Src ip, src port to dest ip, dest port
    \end{itemize}

    \textbf{5.3 Network attacker}

    Unless otherwise stated, MITM can sniff, spoof, modify, drop the header and payload

    MITM in layer x means MITM along the layer x virtual connection (MITM can see and modify data unit of that layer)

    \begin{enumerate}
        \item MITM in the physical layer
        \begin{itemize}
            \item Tap into the internet cable, sniff the wireless communication in the cafe
        \end{itemize}
        \item MITM in the link layer
        \begin{itemize}
            \item Malicious cafe owner who offers the wifi
        \end{itemize}
        \item MITM in the IP / Transport layer
        \begin{itemize}
            \item ISP (e.g. Singtel)
        \end{itemize}
        \item MITM in the application
        \begin{itemize}
            \item Malware in the browser
        \end{itemize}
    \end{enumerate}

    \textbf{5.4 DNS attacks}

    \textbf{DNS query and answer}

    \begin{itemize}
        \item Client sends a query to DNS server using UDP
        \item DNS sends the answer back using UDP
        \item The query contains a 16-bit number known as QID (query ID)
        \item The response from the server must also contain a QID
        \item If the QID in the response does not match the QID in the query, the client
        rejects the answer
    \end{itemize}

    \textbf{DNS spoofing}

    \begin{itemize}
        \item Alice is using free cafe wifi and wants to visit and log in to \texttt{comp.nus.edu.sg}
        \item Consider an attacker who is also in the cafe. Since the wifi is not protected, the attacker can
        \begin{itemize}
            \item Sniff data from the communication channel
            \item Inject spoofed data into the communication channel
        \end{itemize}
        \item The attacker cannot remove/modify data sent by Alice
        \item Attacker owns a webserver which is a spoofed NUS website
    \end{itemize}


    \begin{Figure}
        \centering
        \includegraphics[width=\linewidth]{dns_spoofing.jpg}
    \end{Figure}

    Cannot verify if response is coming from correct source or has not been modified - only matches QID

    If cached into local DNS server (should have expiry), Alice will go to the fake website
    all the time

    \textbf{Denial of Service on DNS}

    \begin{itemize}
        \item A DNS server an be a ``single-point-of-failure'' of the network
        \item DoS attacks can be launched on a web service instead of directly attacking the web server
        \item When DNS server is down, the web service is no longer reachable
        \item Countermeasures: redundant servers, rate limiting, etc.
    \end{itemize}

    \textbf{5.5 Poisoning ARP table}

    \textbf{Address resolution protocol (ARP)}

    \begin{itemize}
        \item Resolution of IP address to mac address
        \item Data link layer
        \item When a device knows the IP address of the next hop router on the same network but
        needs the corresponding MAC address
        \item ARP resolves the router's IP address ot its MAC address to allow packet forwarding at
        the data link layer
    \end{itemize}

    \textbf{Switch}

    \begin{itemize}
        \item Directs data packets between devices or nodes on teh same local network using MAC 
        addresses
        \item The switch keeps a table that associates the port to the mac addresses
        \item Switch does not understand and does not store IP addresses.
    \end{itemize}

    \textbf{Address resolution Protocol (ARP) Table}
    
    \begin{itemize}
        \item An ARP table is a database maintained by each device or nodes on a subnet
        \begin{itemize}
            \item Stores mappings between IP addresses and MAC addresses
        \end{itemize}
    \end{itemize}

    Eg. if N2 wants to send to IP addr 10.0.1.4, 
    \begin{itemize}
        \item N2 looks up ARP table to find MAC addr
        \item N2 sends frame to swtich, specifying destination MAC addr
        \item Switch looks up table to find port
        \item Switch redirects frame to correct port
    \end{itemize}

    If ARP table does not have info of a particular IP addr, N2 will broadcast an ARP request packet
    to all devices on the local network ``Who has IP address X.X.X.X? Tell me your MAC addr''

    The device with the requested IP address recieves the ARP request and replies with an ARP
    reply packet. He sends the IP address and MAC address over. Then the requesting node will update
    its ARP table with the new IP-to-MAC address mapping

    Any node can also broadcast the info or to a specific node even if there isn't a request

    \textbf{Attack: N1 wants to be MITM}

    \begin{itemize}
        \item N1 informs N2 that N3's MAC address is N1's
        \item N1 informs N3 that N2's MAC address is N1's
        \item ARP tables of N2 and N3 are now poisoned
        \item All the frames will be sent to N1, and N1 can relay the frames, or modify the frames before relaying
        \item If N1 does not forward, then it is DoS.
    \end{itemize}

    \textbf{DoS Attacks}

    \begin{itemize}
        \item Attempt to disrupt the normal functioning of a targeted server, service or network
        $\Rightarrow$ affect availability
        \item Many successful DoS attacks simply fllood the victims with overwhelming requests/data
        \item DoS carried out by a large number of attackers is called DDoS (distributed denial of service)
    \end{itemize}

    \textbf{Reflection Attack}
    \begin{itemize}
        \item A type of DoS in which the attackers send requests to intermediate nodes, which in turn 
            send overwhelming traffic to the victim.
        \item Attacker spoofs the victim's IP address as the source
        \item Intermediate nodes then send responses back to the spoofed IP (the victim)
        \item Indirect $\Rightarrow$ more difficult to trace
        \item Preventive measures: 
        \begin{itemize}
            \item Most routers are configured to not broadcast by default
            \item Configure firewalls to block incoming ICMP echo requests packets
            directed at broadcast addresses
        \end{itemize}
    \end{itemize}

    \textbf{Amplification Attack}
    \begin{itemize}
        \item Variation of reflection attacks where the intermediate nodes response is significantly
        larger than the attacker's request
        \item A single request could trigger multiple responses from the intermediate nodes
    \end{itemize}

    \textbf{5.7 Securing different layers}

    \textbf{SSL / TLS}
    \begin{itemize}
        \item Between layers 4 and 5
        \item On top of transport layer
        \item When an application wants to send data to the other end point, it first passes the data
        and the address to SSL / TLS
        \item SSL / TLS then protects the data using encryption and mac
    \end{itemize}

    \textbf{IPSec}
    \begin{itemize}
        \item Layer 3
        \item Mechanism that protects the IP layer and secures all IP traffic between endpoints
        \item Securing network connections between host-to-host, network-to-network (gateways), or
        network-to-host (gateway and host)
    \end{itemize}

    \textbf{WPA2 (wifi protected access 2)}
    \begin{itemize}
        \item Layers 1 and 2
        \item Protect data transmitted over wifi networks
        \item Offers protection in layer 2 and layer 2
        \item Data travelling between a wireless device and the access point is confidential
        and protected from eavesdropping
        \item Not all information in layer 2 are protected
    \end{itemize}

    \textbf{VPN (virtual private network)}
    \begin{itemize}
        \item Tunnel at layer 3
        \item Enable remote user to securely connect to private network
        \item VPN client and VPN server establish a connection, called a tunnel. After authentication
        and verification, establish session keys
        \item When Alice communicates with Bob, VPN client encrypts the entire payload and adds a new IP
        header
        \item From Bob's POV, Alice is communicating with the virtual node with IP address in NUS and does not know
            Alice's actual IP address
    \end{itemize}


    \textbf{Which layer to protect}

    \begin{itemize}
        \item By protecting lowest layer, can protect info in all layer but not feasible.
        \item Intermediate node need to access some higher layer info and sit in higher layer. Hence
        malicious intermediate node could be a MITM in higher layer.
        \item TLS / SSL + WPA2 
        \item IPSec is expensive to implement and difficult to deploy
    \end{itemize}

    \textbf{5.6 Firewalls and IDS}

    In a computer network in an organization, some nodes contain more sensitive information than other. Some nodes
    are more ``secure'' (certain nodes operate in a more hostile environment). We need to divide the 
    network into segments and deny unnecessary access.
    \begin{itemize}
        \item Principle of least privilege: control access to the network
        \item Compartmentalization: keep things separated to limit the impact of any single failure
            or attack
    \end{itemize}
    Tools to control access to the network: firewall , IDS (intrusion detection system)

    \textbf{Firewall}: Gatekeeper, prevents unauthorized access

    \textbf{IDS}: Watcher that raises alerts by monitoring and analysing

    \textbf{Firewall and DMZ}

    \begin{itemize}
        \item A firewall controls what traffic is allowed to enter the network (ingress filtering) or leave
        the network (egress filtering)
        \begin{itemize}
            \item Firewall are devices or programs that control the flow of network traffic between networks or hosts that 
            employ differing security postures.
        \end{itemize}
        \item DMZ: Demilitarized zone
        \begin{itemize}
            \item A sub-network that exposes the organization's external service to the (untrusted) 
            internet.
            \item Separates an internal local area network (LAN) from untrusted external networks, usually
            the internet.
            \item Adds an extra layer of security to an organization's internal network.
        \end{itemize}
        \item DMZs are created using firewalls
    \end{itemize}

    \textbf{2-firewall setting}

    \begin{Figure}
        \centering
        \includegraphics[width=\linewidth]{2_firewall.png}        
    \end{Figure}

    If the web / mail server in DMZ is compromised, the attacker still has to bypass
    the back-end firewall to reach sensitive data.

    One firewall restricts external access, the other restricts access to the internal network

    \textbf{Packet filtering / screening}

    \begin{itemize}
        \item Firewall's controls are achieved by ``packets filtering''
        \item Packet filtering inspects every packet, typically only on the TCP / IP packet's 
        header information (network \& transport layer).
        \item If the payload is inspected, we call it deep packet inspection (DPI).
        \item Actions: Allow to pass, drop, reject, log, notify admin, modify
    \end{itemize}

    \textbf{Types of Firewall}

    \begin{enumerate}
        \item Packet filters
        \begin{itemize}
            \item Inspect only header (mainly IP packet's header)
            \item Use: blocking traffic from certain IP or port
        \end{itemize}
        \item Stateful inspection
        \begin{itemize}
            \item Keep a state on previously received packets
            \item E.g. counting number of connection a particular IP address has made in the past one hour
            \item Use: blocking abnormal connection pattern or unauthorized session attempts
        \end{itemize}
        \item Proxy
        \begin{itemize}
            \item Act as intermediaries that fully receive inspect, and forward (possibly modify) packets between
            client and server
            \item Use: block certain URLs or scan for malware in HTTP traffic
        \end{itemize}
    \end{enumerate}

    \textbf{Intrusion Detection System (IDS)}
    \begin{itemize}
        \item An IDS is a security tool or software that monitors computer systems and networks for signs of 
        malicious activity, policy violations, or security breaches.
        \item An IDS system consists of a set of ``sensors'' that gather data such as logs, network packets, etc.
        Sensors can be deployed on hosts, or netowrk router.
        \item Data are analyzed for intrusion either in real-time or after collection to detect suspicious
        patterns, attacks, or abnormal behaviour.
    \end{itemize}

    \textbf{Three types of IDS}

    \begin{itemize}
        \item Attack Signature Detection
        \begin{itemize}
            \item Looks for specific, well-defined patterns or ``signatures'' of known attacks in the data
            collected by sensors (e.g. using certain port number / source ip addr)
        \end{itemize}
        \item Abnormaly Detection
        \begin{itemize}
            \item IDS attempts to detect abnormal patterns that deviate from the established ``normal'' behaviour of
            the network (e.g. sudden surge of packets with certain port number)
        \end{itemize}
        \item Behavior-based IDS
        \begin{itemize}
            \item Anomaly detection that focuses on human behaviour (e.g. system keeps the profile of
            each user and detects any user who deviates from the profile)
        \end{itemize}
    \end{itemize}

    \textbf{\uline{6. Web security}}

    \textbf{6.1 Background}

    Threat model 1: attackers as another end system

    \begin{Figure}
        \centering
        \includegraphics[width=\linewidth]{web_threat_1.png}        
    \end{Figure}

    Threat model 2: attackers as MITM

    \begin{Figure}
        \centering
        \includegraphics[width=\linewidth]{web_threat_2.png}        
    \end{Figure}

    \begin{itemize}
        \item Attacker is MITM in the IP or lower layer 
        \item Can gain MITM in a few ways
        \begin{itemize}
            \item Cafe owner providing free wifi
            \item DNS spoofing attack or ARP attack
            \item Owner of the VPN server, last hop in TOR network
        \end{itemize}
    \end{itemize}

    \textbf{6.3 Misleading user on domain name}

    \begin{itemize}
        \item Hostname could contain character that resembles ``/''
        \item Address bar spoofing
        \begin{itemize}
            \item Address bar is the only indicator of which url the page is rendering
            \item If the address bar can be ``modified'', attacker can trick the user to visit a malicious url X,
            but user thinks that the url is Y.
            \item Poorly-designed browser can allow attacker to achieve that (e.g. Last time malicious page could overlay
            spoofed address bar on top of the actual bar. Now cannot anymore)
        \end{itemize}
    \end{itemize}

    \textbf{6.4 Cookie and same-origin policy}

    The script in a web page A can access cookies stored by another web page
    B only if both A and B have the same origin (protocol, hostname, port number)

    Application of cookie: token-based authentication.

    \begin{Figure}
        \centering
        \includegraphics[width=\linewidth]{token_based_auth.png}        
    \end{Figure}

    \textbf{Choice of token}

    If token $t$ is a random number, server needs a db to store all issued tokens. 
    Instead, can use
    \begin{itemize}
        \item (secure) MAC. Use random value or some meaningful information like expiry date, concat w mac
        computed using the server secret key
        \begin{itemize}
            \item Relies on security of mac
        \end{itemize}
        \item (insecure) Some meaningful info concat w a sequence number that can be predicted
        \begin{itemize}
            \item Relies on obscurity. If attacker knows how it is generated, he can forge it
        \end{itemize}
    \end{itemize}

    \textbf{6.5 XSS (cross site scripting) attacks}

    The one where u put the \texttt{<script></script>}

    In some websites, if the browser sends a text containing a substring $s$, then replying html sent
    by server would contain $s$. Can add a script in $s$.

    \textbf{Attack}

    \begin{enumerate}
        \item Attacker tricks a user to cick on a url, which contains the target website, and 
        a malicious script $s$.
        \item The request is sent to the server.
        \item The server constructs a response html. The response contains the script $s$.
        \item Browser renders the html page and runs $s$.
        \item Malicious script could deface the original webpage or steal cookie.
    \end{enumerate}

    The attack exploits the client's trust of the server

    \textbf{Types of XSS}

    \begin{itemize}
        \item Reflection (non-persistent)
        \item Stored XSS (persistent). Script $s$ is stored in the targeted website (e.g. in forum page)
    \end{itemize}

    \textbf{Defense}

    Input validation carried out by server

    \textbf{XSRF (cross site request forgery / sea surf)}

    Reverse of XSS. Exploits server's trust of client. 
    
    Example:
    \begin{enumerate}
        \item Suppose client A is already authenticated by a targeted website \texttt{www.bank.com} and
        the site keeps a cookie as ``token''.
        \item Attacker B tricks A to click on a url of S. The url maliciously requests for a service, e.g. transferring
        \$1000 to Bob's account
        \item Since cookie sent to S, A is already authenticated and transaction will be carried out.
    \end{enumerate}

    \textbf{Defense}

    Include authentication information in request as param in url

    \textbf{6.7 Other attacks}

    Web tracking, drive-by-download, pixel stealing, clickjacking, user interface redress attack, CAPTCHA, click fraud

    \textbf{Common simple implementation mistakes}

    Client side authentication / filtering, security credential embedded in public web pages,
    sever's secrets stored in cookies, configuration errors, URL as secrets (e.g. in password reset link or zoom link)


    \textbf{\uline{8. Secure programing}}

    Program must be correct, efficient, secure

    \textbf{8.1 Unsafe function }\texttt{printf()}

    \begin{itemize}
        \item \texttt{printf(format, s)}
        \item e.g. \texttt{printf("the value in temp is \%d\char`\\n", temp)}
    \end{itemize}

    \textbf{Example 1}

    \texttt{printf("hello world \%d")} $\Rightarrow$ \texttt{printf()} will still fetch value of 
    2nd param from the supposing location and display it $\Rightarrow$ unknowingly revealed extra info to the person observing the screen. Confidentiality
    compromised.

    \textbf{Example 2}
    \vspace{1pt}
    \begin{verbatim}
        int main() {
            char t[100];

            scanf("%s", t);
            printf(t);
        }
    \end{verbatim}

    User can set \texttt{t} to be ``\texttt{hello world \%d}'', then can get info

    \textbf{Preventive measure}

    Avoid: 

    \begin{itemize}
        \item \texttt{printf(t)}
        \item \texttt{printf(t, a1, a2)}
    \end{itemize}

    Use:
    
    \begin{itemize}
        \item \texttt{printf("hello");}
        \item \texttt{printf("The value of \%s is \%d", a1, a2);}
    \end{itemize}

    \textbf{How it can be exploited}
    \begin{enumerate}
        \item Obtain more information (confidentiality)
        \item Cause the program to crash (execution integrity)
        \item Modify the memory content using \texttt{"\%n"} (memory integrity which might lead to execution 
        integrity)
    \end{enumerate}

    If the program that invokes \texttt{printf()} has elevated privilege, a user might be able to obtain
    information that was previously inaccessible. 

    \textbf{8.2 Data Representation}

    \textbf{Example 1: string representations}

    \begin{itemize}
        \item \texttt{printf()} uses null termination. Length is not stored. 
        \item Some systems use non-null termination.
        \item Systems that use both null and non-null definitions to verify the certificate may get confused
        \begin{itemize}
            \item E.g. refer to 4.3 for when browser verfies cert based on non-null but displays name based on null.
        \end{itemize}
    \end{itemize}

    \textbf{Example 2: IP address}

    IP address can be represented as
    \begin{itemize}
        \item String e.g. ``132.127.8.16''
        \item 4 integers, and each is a 32-bit integer
        \item A single 32-bit integer
        \item etc.
    \end{itemize}

    E.g. A blacklist is stored in 4 arrays of integers \texttt{A, B, C, D}. Function \texttt{BL}
    takes in 4 integers \texttt{a, b, c, d} and check if the IP address represented is in the
    blacklist. It searches for the index \texttt{i} s.t. \texttt{A[i] == a, B[i] == b, C[i] == c, D[i] == d}

    Suppose another program that uses BL is written using the following flow:
    \begin{enumerate}
        \item Get a string \texttt{s} from user.
        \item Extracts four integers in this way:
        \item \texttt{a, b, c, d = int(s.split("."))} where int converts to 32-bit int
        \item Invokes \texttt{BL(a, b, c, d)}. If TRUE, quits
        \item Else, let $\texttt{ip} = a \times 2^{24} + b \times 2^{16} + c \times 2^8 + d$ where \texttt{ip} is a 32-bit int
        \item Continue the rest of the processing with address \texttt{ip}
    \end{enumerate}

    The above program is vulnerable because if ip address of 11.12.1.0 is blacklisted, user can change input string
    to ``11.12.0.256'', then a, b, c, d = 11, 12, 0, 256 and not detected by \texttt{BL} but ip becomes 11.12.1.0.

    To prevent this, \textbf{use canonical representation} by converting to a standard representation immediately. Do not trust
    input from user.

    \textbf{8.3 Buffer overflow}

    C and C++ do not employ ``bound check'' during runtime. Efficient but prone to bugs.

    \begin{verbatim}
        int a[5]; // Size 5, up to index 4
        int b;
        int main() {
            b = 0;
            printf("value of b is %d\n", b);
            a[5] = 3;
            printf("value of b is %d\n", b);
        }
    \end{verbatim}

    Value 3 is written to \texttt{a[5]}, which is also the location of \texttt{b}

    \texttt{strcpy} is also prone to buffer overflow. Use \texttt{strncpy(s1, s2, n)} instead so that at most \texttt{n}
    chars are copied

    \textbf{Heartbleed}

    Heartbleed is a protocol for 2 connecting entities to check whether the connection has broken.

    A to B: If u r alive, repeat after me this \texttt{x}-character string \texttt{s}.

    B to A: \texttt{s}.

    OpenSSL library didn't implement it securely and didn't verify that the length of the string \texttt{s} is \texttt{x}.
    E.g. if \texttt{x = 500} but \texttt{s = "POTATO"}, B will output 500 characters starting from the location of \texttt{s} in 
    B's memory

    \textbf{Stack smashing}

    Special case of buffer overflow that targets stack. Called stack overflow, stack overrun, stack smashing.

    In call stack, if return address is modified, execution control flow will be changed. Can inject attacker's shell-code 
    into the memory, then run the shell-code.

    \textbf{8.4 Integer Overflow}
    
    \texttt{a = 254; a += 2} will give \texttt{a = 0}

    The predicate that \texttt{a < a + 1} is not always true.

    \textbf{8.5 Code (script) injection}

    \textbf{SQL injection attack}

    \texttt{SELECT * FROM client WHERE name = `\$userinput'}

    User can set input to \texttt{anything' OR 1=1 --}

    \textbf{Prompt injection}

    If teacher use LLM to mark scripts, students can add prompts in pdf file with text same color as background.
    LLM is confused between data and instruction.

    \textbf{8.6 Undocumented access point (easter eggs)}

    Programmers insert undocumented access points to inspect states for debugging purposes. E.g. by pressing certain combi of keys
    , value of certain variables would be displayed, or for certain input str the program branches to debuggingmode

    If these access points mistakenly remain in the final production system, it provides a ``back door'' for attackers.

    \textbf{8.7 Race Condition (TOCTOU)}

    \textbf{TOCTOU (time-of-check-time-of-use)} is a race condition in the context of security. 
    
    \begin{enumerate}
        \item Process A checks the permission to access the data, followed by accessing the data
        \item Process B (malicious) swaps the data
    \end{enumerate}

    If B manages to swap the data between time-of-check and time-of-use in A, then the attack succeeds.

    \textbf{8.8 Defense and preventive measures}

    \textbf{Input validatation, filtering, parameterized queries (SQL)}

    Perform input validation whenever input is obtained from user. If not in expected format then reject. Can be white list or black list.
    For both white and black list, no assurance that all malicious input will be blocked.

    Difficult to design a filter that is complete (doesn't miss out any malicious string) and accepts all legitimate inputs

    \textbf{Parameterized queries}
    \begin{itemize}
        \item Mechanisms introduced in some SQL servers to protect against SQL injection. Queries sent are divided into
        queries and parameters.
        \item SQL parser will know that the parameters are ``data'' and not ``script''.
        \item SQL parser is designed to never execute any script in the parameters.
        \item Still cannot stop XSS.
    \end{itemize}

    \begin{verbatim}
        sqlQuery = `SELECT * FROM custTable WHERE
                    User=? AND Pass=?'

        parameters.add("User", username)
        parameters.add("Pass", password)
    \end{verbatim}

    \textbf{Use ``safe'' function}

    Use safe versions of functions that are known to create problems, e.g. \texttt{strncpy} instead of \texttt{strcpy}

    But still can be vulnerable e.g. one uses a combination of \texttt{strlen()} and \texttt{strncpy()}

    \textbf{Bound checks}

    Some programming languages perform bound checking during runtime by storing upper and lower bounds during array instantiation
    so for \texttt{a[i] = 5;}, check if i $<$ lower bound or i $>$ upper bound then stop else assign 5 to location

    \textbf{Type safety}

    Some programming languages carry out type checking to ensure that the arguments an operation gets during execution
    are always correct. Can be done during runtime (dynamic type check) or when it is being compiled (static type check).

    \textbf{Canaries}
    
    \begin{itemize}
        \item Canaries are secrets inserted at carefully selected memory locations during runtime
        \item Program halts if values are modified
        \item Helps to detect overflow esp. stakc overflow bc if attacker wants to write to a particular
        memory location via overflow, canaries would be modified.
        \item But value needs to be kept secret else attacker can write secret value to canary
        while over-running it
    \end{itemize}

    \textbf{Memory randomization}

    Attacker is at an advantage when data and codes are always stored in the same locations. Address space layout
    randomization (ASLR) can help to decrease the attacker's chance of success.

    \textbf{Code inspection}
    
    \begin{itemize}
        \item Manual checking (tedious)
        \item Automated checking
        \begin{itemize}
            \item Taint analysis: variables that contain input from the (potentially malicious) users are labeled
            as source. Critical functions are labeled as sink. Taint analysis checks whether the sink's arguments could
            potentially be affected (i.e. tainted) by teh source. If so, special check (for e.g. manual inspection) would
            be carried out. The taint analysis can be static (i.e. checking the code without ``tracing it'') or dynamic (i.e. run the code with some input).
            \item E.g. \textbf{Sources:} user input, \textbf{Sink:} opening of system files, function that evaluates a SQL command, etc.
        \end{itemize}
    \end{itemize}

    \textbf{Testing}

    \begin{itemize}
        \item Types:
        \begin{itemize}
            \item White-box testing: tester has access to source code
            \item Black-box testing: tester does not have access to source code
            \item Grey-box testing: Combination of the above
        \end{itemize}
        \item Security testing attempts to discover intentional attack, so need to test for inputs that
        will rarely occur under normal circumstances.
        \item Fuzzing is a technique that sends malformed inputs to discover vulnerability. More effective than sending in random input.
        Fuzzing can be automated or semi-automated.
    \end{itemize}


    \textbf{Principle of Least Privilege}

    The principle of least privilege (PoLP), also known as the principle of minimal privilege (PoMP) or the principle of
    least authority (PoLA), requires that in a particular abstraction layer of a computing environment, every module (such
    as a process, a user, or a program, depending on the subject) must be able to access only the information and
    resources that are necessary for its legitimate purpose.

    \begin{itemize}
        \item E.g. When deploying a software system, do not grant users more access rights than
        necessary, and avoid enabling unnecessary options.
        \begin{itemize}
            \item For instance, a webcam application might offer various functions that allow users to control the device
            remotely. Typically, users can choose which features to enable or disable. As the software developer, you
            should consider whether all features should be turned on by default when the product is delivered to clients.
            If every feature is enabled by default, it becomes the client’s responsibility to disable those that are
            unnecessary. However, clients may not fully understand the security implications, which can increase their
            risk exposure.
        \end{itemize}
        \item E.g. in Canvas, consider the appropriate level of access to grant a student TA. If the TA’s role
        does not require editing quizzes, they should not be given permission to modify them.
    \end{itemize}

    \textbf{Patching}

    Life cycle of vulnerability:
    Vulnerability is discovered $\rightarrow$ affected code is fixed $\rightarrow$ revised version is tested
    $\rightarrow$ patch is made public $\rightarrow$ patch is applied

    \begin{itemize}
        \item Patch can be useful to attackers bc attackers can inspect the patch and derive the vulnerability
        \item Number of successsful attacks goes up after vulnerability / patch is announced as more attackers are
        aware of the exploit
    \end{itemize}

    \begin{Figure}
        \centering
        \includegraphics[width=0.6\linewidth]{vulnerability_life_cycle.png}        
    \end{Figure}

    \begin{itemize}
        \item Need to apply patch timely
        \item For critical system, not wise to apply before rigorous testing
    \end{itemize}

    \textbf{9. Access Control}

    \textbf{9.1 Access Control model}

    Access control: controlling operations on objects by subjects

    \textbf{Security perimeter}

    \begin{itemize}
        \item Malicious activities outside of the boundary would not affect resources within
        the perimeter.
        \item Malicious activities withinn the boundary stays within the boundary.
        \item Design of the boundary is guided by
        \begin{itemize}
            \item Principle of least privilege
            \item Compartmentalization
            \item Defense in depth / swiss cheese model
            \item Segregation of duties
        \end{itemize}
        \item Examples:
        \begin{itemize}
            \item Calculator app should not have access to contacts so that even if the app is malicious / vulnerable
            , the confidentiality / integrity of contacts are still preserved (Principle of Least Privilege)
            \item School website hosts two services: (1) course's fee payment and (2) exam result. Perimeter between them so that
            result system remains intact even if got SQL injection attack on (1). (Compartmentalization)
            \item A company deploys a firewall separating their server from DMZ. An IDS (intrusion Detection System) reside in
            the firewall to detect malicious traffic based on known attack signature. In addition, processes in the server are
            monitored for abnormal behavior. Attack that evade the firewall might be caught by the process monitor, and
            vice versa. (Defence in depth) (Swiss Cheese Model)
            \item A company keeps backup of is business-critical data. The company implements a policy: a single person must
            not have access to both the production copy and the backup copy. Assigning different components to different
            person is aka Segregation of Duties. The goal is to eliminate single-point-of-failure. With that, a single rogue
            system admin (insider) is unable to corrupt all. (Segregation of duties).
        \end{itemize}
    \end{itemize}

    \textbf{Security perimeter on Android}

    Android apps must request permission to access sensitive user data such as contacts and SMS, as well as certain
    system features (such as camera and internet). 

    A central design point of th Android security architecture is that no app, by default, has permission to perform any
    operations that would adversely impact other apps, the OS, or the user. This includes reading or writing the user's
    private data (such as contacts or emails), reading or writing another app's files, performing network access,
    keeping the device awake, and so on.

    Each app has a ``manifest'' file which lists down permissions the app wishes to have. During runtime, OS will grant
    the request based on default setting or prompt the user.

    This is different from a typical multi-user system which has no boundary between two apps run by the same user.

    \textbf{Implications}

    \begin{itemize}
        \item A game $G$ and an image editing tool $T$ is implemented for Windows and Android. Alice installed both $G$
        and $T$ in a Windows desktop and Android device. 
        \item $T$ can read / write files generated by $G$ in Windows but not Android
        \item When $G$ is executing $T$ cannot access the memory space of $G$ for both Android and Windows
        \item $T$ cannot modify the installation of $G$ in Android but it can in early versions of Windows
        \item In android / ios, information is passed from one app to another only when the user explicitly gives permission
        by indicating in ``share''.
    \end{itemize}

    Newer verisions of OS like Windows and Mac have started to impose boundaries between apps

    \textbf{Principal / subject, operation, object}
    
    A principal (or subject) wants to access an object with some operation. The reference monitor 
    either grants or denies the access.

    E.g. In Canvas, a \textbf{Student} wants to \textbf{submit} a \textbf{forum post}

    \begin{itemize}
        \item Principals: human users
        \item Subjects: Entities in the system that operate on behalf of the principals
    \end{itemize}

    Accesses to objects:
    \begin{itemize}
        \item Observe: e.g. reading a file
        \item Alter: e.g. writing a file, deleting a file, changing properties
        \item Action: e.g. executing a program
    \end{itemize}

    \textbf{Definition: ownership}

    Every object has an owner. Access rights to an object are decided by:
    
    \begin{enumerate}
        \item Owner of the object decides the rights (discretionary acces control)
        \item System-wide policy decides (mandatory access control)
    \end{enumerate}

    \textbf{9.2 Access Control Matrix}

    \begin{Figure}
        \centering
        \includegraphics[width=0.8\linewidth]{access_control_matrix.png}
    \end{Figure}
    r: read, w: write, x: execute, s: execute as owner, o: owner

    Seldom explicitly stored bc the table is very large and difficult to manage

    \textbf{Access Control List (ACL) and Capabilities}

    Access control matrix can be represented as ACL or capabilities

    \textbf{ACL}

    \begin{Figure}
        \centering
        \includegraphics[width=0.8\linewidth]{ACL.png}        
    \end{Figure}

    \textbf{Capability}
    \begin{Figure}
        \centering
        \includegraphics[width=\linewidth]{capability.png}        
    \end{Figure}

    Most Unix file systems use ACL. 

    \textbf{9.3 Intermediate Control}

    We want an intermediate control that is fine grain (e.g. in Facebook, allow user to specify which
    friend can view a particular photo) and yet easy to manage.

    Not practical for owner to specify each single entry in the access control matrix, so we ``group''
    subjects / objects and define the access rights n the group. This is called intermediate control.

    \begin{Figure}
        \centering
        \includegraphics[width=0.9\linewidth]{grouping.png}        
    \end{Figure}

    Role-based access control

    \textbf{Protection rings}

    In OS, ``privilege'' is often called protection rings. Each object (data) and subject (process) is asssigned
    a number. Objects with smaller numbers are more important. We call processes with lower ring number as having “higher privilege”.
    A subject cannot access (both read/write) an object with smaller ring number. 

    Unix only has 2 rings: superuser and user

    \textbf{Bell-LaPadula vs Biba}

    \textbf{Bell-LaPadula}

    \begin{itemize}
        \item No read up: Prevents lower level from getting info in higher level
        \item No write down: Prevents malicious insider from passing information to lower levels
        \item Confidentiality
        \item No ``sensitive'' information leaking down
    \end{itemize}

    \begin{Figure}
        \centering
        \includegraphics[width=0.9\linewidth]{bell_lapadula.png}        
    \end{Figure}

    \textbf{Biba}

    \begin{itemize}
        \item No write up: Prevents a malicious subject from poisonng upper level data
        \item No read down: Prevents a subject from reading data poisoned by lower level subjects
        \item Integrity
        \item No ``malicious'' information going up
    \end{itemize}

    \begin{Figure}
        \centering
        \includegraphics[width=0.9\linewidth]{biba.png}        
    \end{Figure}

    \textbf{9.4 Unix file system}

    \textbf{Unix file system access control}

    Objects in Unix include files, directories, memory devices, and I/O devices

    \texttt{ls -al}

    \texttt{-rw-r--r-- 1 alice staff 124 Mar 9 22:29 my.c}

    \begin{itemize}
        \item First \texttt{-} indicates whether it is a file or directory
        \item Remaining file permissions are grouped into 3 triples that define \textbf{read, write, execute} access for
        \textbf{owner, group, other}
        \item \texttt{1}: links count (not relevant)
        \item \texttt{alice}: owner
        \item \texttt{staff}: group
        \item \texttt{124}: file size
        \item \texttt{Mar 9 22:29}: date \& time of last modification
        \item \texttt{my.c}: filename
    \end{itemize}

    \textbf{Principals, subjects}

    \begin{itemize}
        \item Principals are user-identities (UIDs) and group-identities (GIDs)
        \item Information of user accounts are stored in the ``password'' file \texttt{/etc/passwd}
        \item Subjects are processes and each process has a process ID (PID)
    \end{itemize}

    Now the password file does not actually store the password, last time it did and everyone could see the hashed passwords
    of others, thus it was vulnerable to offline dictionary attack.
    Now it is replaced with \texttt{*} and actual hashed passwords are stored somewhere else.

    \textbf{Superuser (root)}

    UID 0, username root, all security checks are turned off for root

    \textbf{Checking rules for file access}

    \begin{itemize}
        \item The objects are files. Each file is associated with a 9-bit permission.
        \item Each file is owned by a user and a group
        \item Whena user wants to access a file (object), the following are checked in the order:
        \begin{enumerate}
            \item If the user is the owner, the permission bits for owner decide the access rights.
            \item If the user is not the owner, but the user’s group (GID) owns the file, the permission bits for
            group decide the access rights.
            \item If the user is not the owner, nor member of the group that own the file, then the permission
            bits for other decide.
        \end{enumerate}
    \end{itemize}

    Owner of a file, or superuser can change permission bits

    \textbf{9.5 Controlled invocation \& privilege elevation}

    Eg in Unix: Some sensitive resources (such as network port 0 to 1023, printer) should be
    accessible only by the superuser. However, users sometime need those resources.

    Eg: Consider a file F that contains home addresses of all staffs. Clearly, we cannot
    grant any user to read F. However, we must allow a user to read/modify his/her
    address and thus need to make it readable/writeable to that user. The polarized
    setting where either a process can read or cannot read a file would get stuck!

    Solution: \textbf{controlled invocation}.

    \begin{itemize}
        \item The system provides a predefined set of applications that have access to F. 
        \item These applications are granted ``elevated privilege'' so that they can freely
        access the file, and any user can invoke the application. Now, any user can access F via the application.
        \item The programmer who wrote the application bears the responsibility to make sure that the application
        only performs the intended limited operation.
    \end{itemize}

    \textbf{Bridges with elevated privilege}

    \begin{Figure}
        \centering
        \includegraphics[width=\linewidth]{bridge.png}        
    \end{Figure}

    If the bridge is not implemented correctly and contains exploitable vulnerabilities, an attacker
    can trick the bridge to perform ``illegal'' operations not expected by the programmer / designer. This would
    have serious implication, since the process is now running with ``elevated privilege''. 

    Attacks of such form is also known as ``privilege escalation''.

    \textbf{9.6 Controlled invocation in UNIX}

    \begin{itemize}
        \item Process got PID, real UID, and effective UID. 
        \item Real UID is inherited from user who invokes the process.
        \item If Set UID is disabled (permission will be \texttt{`x'}), process' effective UID = real UID
        \item Else (SUID enabled, permission will be \texttt{`s'}), effective UID is inherited from file's owner
    \end{itemize}

    E.g. got file containing personal information of users with SUID disabled, users cannot change their 
    own data, so have another program to edit the file with SUID enabled and owner = root, then anyone can run the program to
    edit their own data.

    \textbf{Tutorials}

    What happens when Alice accesses a website with expired cert?

    \begin{itemize}
        \item Previous owner
        \begin{itemize}
            \item Website may no longer belong to original owner
            \item Previous owner had valid but outdated certificate and can use this against Alice
            \item Previous owner can impersonate the current website
        \end{itemize}
        \item Compromised key
        \begin{itemize}
            \item Private key was stolen after expiry
            \item Certificate was already expired, so website did not revoke certificate
        \end{itemize}
        \item Legacy issue
        \begin{itemize}
            \item SHA1 is broken
            \item Some expired certificates are signed by CA using SHA1
            \item Signature may be forged
            \item Public key may also be short, and thus broken
        \end{itemize}
    \end{itemize}

    \textbf{Renegotiation attack}

    Renegotiation: Update session key without closing TLS session

    \begin{Figure}
        \centering
        \includegraphics[width=\linewidth]{renegotiation_attack.png}        
    \end{Figure}

    In initial traffic, attacker sends:

    \texttt{GET /pizza?toppings=pepperoni;address=attackersaddress HTTP/1.1 X-Ignore-This:}

    In client traffic, genuine client sends:

    \texttt{GET /pizza?toppings=sausage;address=victimssaddress HTTP/1.1 Cookie: victimscookie}

    As such, server receives the following request:

    \texttt{GET /pizza?toppings=pepperoni;address=attackersaddress HTTP/1.1 X-Ignore-This: GET /pizza?toppings=sausage;address=victimssaddress HTTP/1.1 Cookie: victimscookie}

    Client starts an initial negotiation, attacker holds it, starts TLS session with server, sends some prefix
    data to server, then continues negotiation (which from the server's pov is the renegotiation)

    Result: the attacker is able to inject some data into the server buffer before the client traffic comes in.

    \begin{Figure}
        \includegraphics[width=0.3\linewidth]{encrypted_and_authenticated.png}
        
    \end{Figure}

\end{multicols*}
\end{document}