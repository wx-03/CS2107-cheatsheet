\documentclass[landscape,a4paper]{extarticle}
\usepackage{caption}
\usepackage{multicol}
\usepackage[top=1em,bottom=1em,left=1em,right=1em]{geometry}
\usepackage[framemethod=tikz]{mdframed}
\usepackage{microtype}
\usepackage{pdfpages}
\usepackage{amsmath, amssymb, amsthm}
\usepackage{anyfontsize}
\usepackage[shortlabels]{enumitem}
\usepackage{graphicx, float}
\usepackage{ulem} % Using \uline{} instead of \underline{} will make the line closer to the word
\usepackage{xcolor}
\usepackage{tgschola}

\let\bar\overline

% Configure image directory
\graphicspath{{images/}}

\newenvironment{Figure}
  {\par\noindent\minipage{\linewidth}}
  {\endminipage\par\medskip}
  
% Remove caption labels
\captionsetup{labelformat=empty,labelsep=none}

% No paragraph indent
\setlength{\parindent}{0pt}

% No spaces between list items, no left margins
\setlist[enumerate]{nosep, leftmargin=*}
\setlist[itemize]{nosep, leftmargin=*}

% No spaces before and after math mode
\expandafter\def\expandafter\normalsize\expandafter{%
    \normalsize%
    \setlength\abovedisplayskip{0pt}%
    \setlength\belowdisplayskip{0pt}%
    \setlength\abovedisplayshortskip{-8pt}%
    \setlength\belowdisplayshortskip{2pt}%
}

\begin{document}
\fontsize{7}{8}\selectfont
\fontfamily{qcs}\selectfont
\begin{multicols*}{5}
	\textbf{\uline{Topic 0}}

	\textbf{CIA}
	\begin{enumerate}
		\item Confidentiality
		      \begin{itemize}
			      \item Prevention of unauthorized discolusre of information
		      \end{itemize}
		\item Integrity
		      \begin{itemize}
			      \item Prevention of unauthorized modification of information or processes
			      \item Non-repudiation
		      \end{itemize}
		\item Availability
		      \begin{itemize}
			      \item Prevention of unauthorized withholding of information or resources
		      \end{itemize}
	\end{enumerate}
	\textbf{Threat model}

	The description of a class of attacks by:
	\begin{itemize}
		\item The attacker's goals
		\item The attacker's capabilities
	\end{itemize}

	\textbf{Trade-off in security}
	\begin{itemize}
		\item Ease-of-use: Security mechanisms interfere with working patterns
        users were originally familiar with
		\item Performance: Security mechanisms consume more resources and lower
        performance
		\item Cost: Security mechanisms are expensive to develop and manage
	\end{itemize}

	\textbf{Threat-Vulnerability-Control}
	\begin{itemize}
		\item \textbf{Threat}: A set of circumstances that has the potential to cause harm (e.g. an
        attacker with control of the workstation in the LT could maliciously gather
        sensitive info like passwords)
        \item \textbf{Vulnerability}: A weakness in the system (e.g. anyone can reboot the
        workstation from USB or Disk to gain control)
        \item \textbf{Control}: A control, countermeasure, security mechanism is a mean to counter threats
        (e.g. restrict physical access to the workstation, disable USB booting)
        \item \textbf{A threat is blocked by control of a vulnerability}
	\end{itemize}
    
    \textbf{\uline{Topic 1: Encryption}}

    \textbf{1.1 Definition: Encryption/decryption/keys}

    \begin{Figure}
        \centering
        \includegraphics[width=\linewidth]{symmetric_key_encryption.png}
    \end{Figure}

    \begin{itemize}
        \item A symmetric-key encryption scheme consists of encryption and decryption
        \item A cipher must be correct and secure
        \begin{itemize}
            \item \textbf{Correctness}: For any plaintext $x$ and key $k$, $D_k(E_k(x)) = x$
            \item \textbf{Security}: Definition depends on the threat models. Informally,
            from the ciphertext, teh eavesdropper is unable to derive useful information of the
            key $k$ or the plaintext $x$, even if the eavesdropper can probe the system.
        \end{itemize}
        \item Probabilistic encryption: for the same $x$, there could be different $c$'s.
        But they all can be decrypted to the same $x$.
    \end{itemize}


    \textbf{1.2 Security Model and Requirement}
    
    \textbf{Threat model}
    \begin{itemize}
        \item Attacker's goal
        \begin{itemize}
            \item Total break (most difficult goal)
            \begin{itemize}
                \item Attacker wants to find the key
            \end{itemize}
            \item Partial break
            \begin{itemize}
                \item Attacker may want to decrypt a ciphertext but not interested in knowing the key
                \item Attacker may simply want to extract some info abt the plaintext (e.g. if it is a jpg or excel file)
            \end{itemize}
            \item Distinguishability (weakest goal)
            \begin{itemize}
                \item With some non-negligible probability of $>$ 1/2, the attacker acan correctly
                distinguish the ciphertexts of a given plaintext from the ciphertext of another given
                plaintext 
            \end{itemize}
        \end{itemize}
        \item Attacker's capability
        \begin{itemize}
            \item Ciphertext only attack
            \begin{itemize}
                \item Attacker is given a collection of ciphertext $c$. The attacker may 
                know some properties of the plaintext (e.g. the plaintext is an English sentence)
            \end{itemize}
            \item Known plaintext attack
            \begin{itemize}
                \item The attacker is given a collection of plaintext $m$ and their corresponding
                ciphertext $c$
                \item Attacker might get this as they know the header or part of the plaintext
            \end{itemize}
            \item Chosen plaintext attack (CPA)
            \begin{itemize}
                \item The attacker has access to an oracle. The attacker can choose and feed any plaintext
                $m$ to the oracle and obtain the corresponding ciphertext $c$ (all encrypted with
                the same key). The attacker can access the oracle many times, as long as within the attacker's
                compute power. He can see the ciphertext and then choose the next input. This black-box is an
                \textbf{encryption oracle}. 
                \item e.g. attacker has access to a smartcard
                \item e.g. attacker can eavesdrop
            \end{itemize}
            \item Chosen ciphertext attack (CCA2)
            \begin{itemize}
                \item Same as CPA but the attacker chooses the ciphertext and the black-box
                outputs the plaintext. The black-box is a \textbf{decryption oracle}.
                \item Padding oracle is a weaker form of a decryption oracle.
            \end{itemize}
        \end{itemize}
    \end{itemize}

    From defender's POV, want a cipher that can protect against the attacker with the highest
    capability. Cipher is secure against CCA2 (decryption oracle) $\implies$ secure
    against CPA (encryption oracle)

    \textbf{1.3 Classical ciphers + illustration of attacks}

    \textbf{1.3.1 Substitution cipher}

    \begin{itemize}
        \item Plaintext and ciphertext are both strings over a set of symbols $U$.
        \item The key is a 1-1 onto func from $U$ to $U$
        \item Key space: set of all possible keys
        \item Key space size: total number of possible keys
        \item Key size/length: number of bits required to represent a key
        \item Attacks
        \begin{enumerate}
            \item Exhaustive search (examine all possible keys 1 by 1)
            \begin{itemize}
                \item Running time depends on size of key space
                \item If the table size is 27, the key can be represented by a sequence of 27
                symbols. The size of key space is 27!. Exhaustive search eneds to carry out 27!
                loops, which is infeasible using current compute power.
            \end{itemize}
            \item Known plaintext attack
            \begin{itemize}
                \item Given sufficiently long ciphertext, the full table can be found
                \item Substitution cipher is not secure under known plaintext attack.
            \end{itemize}
            \item Ciphertext only attacker
            \begin{itemize}
                \item Given that the attacker knows that the plaintext is an English sentence,
                he can do frequency analysis attack. The frequency of letters used in English is 
                not uniform. Given a sufficiently long ciphertext, attacker may correctly guess the plaintext by 
                mapping frequent characters in the ciphertext to the frequent character in English.
            \end{itemize}
        \end{enumerate}
    \end{itemize}

    \textbf{1.3.2 Permutation cipher}
    \begin{Figure}
        \centering
        \includegraphics[width=\linewidth]{permutation_cipher.png}        
    \end{Figure}
    \begin{itemize}
        \item AKA transposition cipher
        \item First group the plaintext into blocks of $t$ characters, then apply a secret
        permutation to each block by shuffling the characters
        \item The key is the secret permutation, which is a 1-1 onto func $e$ from $\{1, 2, \ldots, t\}$
        to $\{1, 2, \ldots, t\}$. $t$ can also be part of the key.
        \item Attack
        \begin{itemize}
            \item Fails under known-plaintext attack
            \item Easily broken under ciphertext only attack if the plaintext is English text
        \end{itemize}
    \end{itemize}

    \textbf{1.3.3 One Time Pad}

    \textbf{Properties of xor}: 
    \begin{itemize}
        \item Commutative: $A \oplus B = B \oplus A$
        \item Associative: $A \oplus (B \oplus C) = (A \oplus B) \oplus C$
        \item Identity element: $A \oplus 0 = A$
        \item Self-inverse: $A \oplus A = 0$
    \end{itemize}

    \textbf{One Time Pad}
    \begin{itemize}
        \item Encryption: plaintext xor key bit by bit
        \item Decryption: ciphertext xor key bit by bit
        \item Key is only used once, so 1GB of plaintext would need a 1GB key to encrypt
        \item Security
        \begin{itemize}
            \item From a pair of ciphertext and plaintext, attacker can derive the key
            but useless bc key won't be used anymore
        \end{itemize}
    \end{itemize}

    \textbf{1.4 Modern ciphers + recommended key length}

    \textbf{1.4.2 Block cipher \& mode of operations}

    DES/AES are known as block ciphers. Block ciphers have a fixed size of input/output.
    AES: 128 bits (16 bytes). 

    Large plaintext is divided into blocks before applying the block cipher.

    \textbf{ECB (electronic code book) mode}
    \begin{Figure}
        \centering
        \includegraphics[width=\linewidth]{ecb_encryption.png}        
    \end{Figure}
    \begin{Figure}
        \centering
        \includegraphics[width=\linewidth]{ecb_penguin.png}        
    \end{Figure}

    \textbf{CBC (cipher block chaining) mode on AES}
    \begin{itemize}
        \item Initialization vector (IV) is an arbitrary value chosen during encryption, 
        must be different in different encryptions. 
        \item In CBC mode, IV must be unpredictable, else it is susceptable to BEAST attack.
        \item If IV is randomly chosen, it is unpredictable
    \end{itemize}
    \begin{Figure}
        \centering
        \includegraphics[width=\linewidth]{cbc_encryption.png}        
    \end{Figure}
    \begin{Figure}
        \centering
        \includegraphics[width=\linewidth]{cbc_decryption.png}        
    \end{Figure}

    \textbf{CTR (counter) mode}
    \begin{Figure}
        \centering
        \includegraphics[width=\linewidth]{ctr_encryption.png}        
    \end{Figure}
    \begin{Figure}
        \centering
        \includegraphics[width=\linewidth]{ctr_decryption.png}        
    \end{Figure}

    \textbf{GCM mode (Galois/counter)}

    Authenticated encryption, ciphertext consists of extra tag for authentication.
    Secure in the presence of decryption oracle.

    \textbf{1.4.3 Stream cipher and IVs}

    Stream cipher is bit by bit. CTR mode is a "stream cipher" but it is not bit by bit.
    \begin{Figure}
        \centering
        \includegraphics[width=\linewidth]{stream_cipher.png}        
    \end{Figure}

    \begin{itemize}
        \item Need IV and no two IVs can be the same
    \end{itemize}

    \textbf{Stream cipher without IV}
    \begin{Figure}
        \centering
        \includegraphics[width=\linewidth]{stream_cipher_without_iv.png}        
    \end{Figure}

    \textbf{Stream cipher with IV}
    \begin{Figure}
        \centering
        \includegraphics[width=\linewidth]{stream_cipher_with_iv.png}        
    \end{Figure}

    \begin{itemize}
        \item IV makes an encryption probabilistic
    \end{itemize}
    
    \textbf{1.5 Examples of attacks on crypto}

    \textbf{1.5.1 Meet-in-the-middle}

    \begin{itemize}
        \item DES is not secure $\rightarrow$ improve by encrypting multiple times using different
        keys
        \item Consider double encryption under known plaintext attack. Attacker has $m$ and $c$ and
        wants to know $k_1$, $k_2$.
        \item Using exhaustive search, amount of DES encryption/decryption would be $2^{56+56}$
        \item Hence use meet-in-the-middle attack.
        % Compute sets $C$ and $M$, where $C$ contains
        % ciphertexts of $m$ encrypted with all possible keys and $M$ contains plaintexts of $c$ decrypted
        % with all possible keys. Then find all common elements (likely only 1) in $C$ and $M$. From 
        % common elements, obtain the 2 keys.
        \item for $k$-bit keys, this reduces the number of crypto operations to $2^{k + 1}$
    \end{itemize}

    \begin{Figure}
        \centering
        \includegraphics[width=\linewidth]{meet_in_the_middle.png}        
    \end{Figure}

    \textbf{Tradeoff with time and space}

    \begin{Figure}
        \centering
        \includegraphics[width=\linewidth]{meet_in_the_middle_tradeoff.png}        
    \end{Figure}

    \begin{itemize}
        \item Last bit of $k_1$ fixed to 1, last bit of $k_2$ fixed to 0
        \item Perform meet-in-the-middle on the first 2 bits of $k_1$ and $k_2$
    \end{itemize}

    \textbf{1.5.2 Padding Oracle}

    Plaintext needs to be padded to split into blocks

    \begin{itemize}
        \item PKCS\#7 is a padding standard
    \end{itemize}

    \begin{Figure}
        \includegraphics[width=0.5\linewidth]{pkcs7.png}        
    \end{Figure}

    \textbf{Padding oracle attack}

    Attack model: 

    Attacker has:
    \begin{enumerate}
        \item Ciphertext (iv, c) where the ciphertext was encrypted using $k$
        \item Access to a padding oracle
    \end{enumerate}
    Attacker's goal: plaintext of (iv, c)

    Padding oracle input is ciphertext, output is YES if the plaintext is in the correct
    padding format else NO

    \textbf{Padding oracle attack on AES CBC mode}

    Attacker knows:

    \begin{Figure}
        \centering
        \includegraphics[width=\linewidth]{padding_oracle_attacker_knowledge.png}
    \end{Figure}

    \begin{Figure}
        \centering
        \includegraphics[width=\linewidth]{padding_oracle_attack.png}
    \end{Figure}

    \begin{align*}
        iv \oplus d(c) &= 03\\
        iv' \oplus d(c) &= 04
    \end{align*}

    xor the 2 tgt to get $iv' = 07 \oplus iv$

    \begin{align*}
        iv' &= iv \oplus 00\ 00 \ldots t\ 07\ 07\ 07\\
        d(C_5) \oplus t \oplus V_5 &= 04\\
        d(C_5) \oplus V_5 \oplus t &= 04\\
        d(C_5) \oplus V_5 &= x_5\\
        x_5 \oplus t &= 04\\
        x_5 &= 04 \oplus t
    \end{align*}
    
    Keep guessing $t$ until padding oracle outputs YES, then we know $x_5$

    To get next byte:
    \begin{Figure}
        \centering
        \includegraphics[width=\linewidth]{padding_oracle_next_byte.jpg}
    \end{Figure} 

    \textbf{1.6 Pitfalls in usages and implementations}
    \begin{enumerate}
        \item Wrong choice of IV / reusing one-time pad
        \item Randomness is predictable
        \item Modify existing or design your own encryption scheme
        \item Reliance on obscurity: Kerkchoff's principle
        \begin{itemize}
            \item Kerkchoff's principle: A system should be secure ven if everything about the
            system, except the secret key, is public knowledge
        \end{itemize}
    \end{enumerate}

\end{multicols*}
\end{document}